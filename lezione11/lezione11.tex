\documentclass[12pt]{article}
\usepackage[utf8]{inputenc}
\usepackage[margin=1.5cm]{geometry}
\usepackage{amsmath}
\usepackage{amssymb}
\usepackage{amsthm} 
\usepackage{graphicx}
\usepackage{mathtools}
\usepackage[normalem]{ulem}
\newtheorem*{esempio}{Esempio}
\DeclarePairedDelimiter{\abs}{\lvert}{\rvert}
\newcommand{\inter}{\begin{matrix}\prod\end{matrix}}

\begin{document}

\section[Lezione 11 - Iterazioni di punto fisso]{Lezione 11 - Iterazioni di punto fisso, teorema delle contrazioni, convergenza locale, ordine di convergenza, Newton come iterazione di punto fisso}

In questa lezione studieremo equazioni della forma 
\[ x=\phi(x), \ x \in I \subseteq \mathbb{R} \]
dove $\phi \in C(I)$ e $I$ è un intervallo chiuso (non necessariamente limitato) di $\mathbb{R}$ e la loro soluzione numerica tramite semplici iterazioni del tipo
\[ x_{n+1} = \phi(x_n), \ n \geq 0, \ x_0 \in I \]
comunemente dette ``iterazioni di punto fisso".\\
In particolare, vedremo ipotesi che garantiscano la convergenza $x_n \rightarrow \xi$, $n \rightarrow \infty$ dove $\xi = \phi(\xi)$ è detto punto fisso di $\phi$ in $I$ e studieremo l'ordine di convergenza dell'iterazione, scoprendo che il metodo di Newton  può essere interpretato come iterazione di punto fisso. \\
Enunciamo qui sotto un famoso teorema, detto ``teorema delle contrazioni".

\subsection{TEOREMA (esistenza e unicità del punto fisso e convergenza delle iterazioni di punto fisso per una \uline{contrazione})}
\begin{center}
    \fbox{\begin{minipage}[t]{15cm}%
        Sia $\phi:I\subseteq\mathbb{R}\rightarrow\mathbb{R}$ una funzione derivabile nell'intervallo chiuso $I\subseteq\mathbb{R}$, tale che:
        \begin{enumerate}
            \item $\phi(I)\subseteq I$ cioè l'immagine di $I$ tramite $\phi$, $\phi(I)=\left\{y:y=\phi(x), \ x\in I\right\}$, è contenuta in $I$
            
            \item $\exists\,\theta\in(0,1):\abs{\phi'(x)}\leq\theta \ \ \forall x\in I$ 
        \end{enumerate}
        allora 
        \[\exists! \,\xi \in I:\xi = \phi(\xi) \text{ (punto fisso) } \forall x_0 \in I, \ \xi=\lim_{n\to\infty} x_n\]
        \[\text{dove } x_{n+1}=\phi(x_n), \ n\geq0\]
    \end{minipage}}
\end{center}
Prima di dimostrare questo teorema (che può essere esteso ad ambiti molto più astratti (qui ci limitiamo a funzioni reali di variabile reale) come provato dal matematico polacco S. Banach nel 1919 facendolo diventare uno dei risultati chiave dell'analisi matematica contemporanea), facciamo alcune osservazioni:
\begin{enumerate}
    \item l'intervallo $I$ è assunto chiuso, ma può non essere limitato, cioè $I=[a,b]$ con $-\infty < a < b < +\infty$ ma anche $I=[a, +\infty)$ oppure $I = (-\infty,b]$ o addirittura $I=\mathbb{R}$
    \item $\phi$ è una contrazione (di $I$ in sé stesso), cioè contrae le distanze di un fattore $\theta < 1$. Infatti per il teorema del valor medio $\forall x,y \in I$
    \[ \phi(x)-\phi(y)=\phi'(z)(x-y), \quad z \in int(x,y) \]
    da cui
    \[ \abs{\phi(x)-\phi(y)}=\abs{\phi'(z)}\abs{x-y}\leq\theta\abs{x-y}<\abs{x-y}\]
    \item chiaramente la disuguaglianza appena provata, assunta direttamente come ipotesi, implica che $\phi$ è continua in $I$, infatti $\forall x,\overline{x}\in I$
    \[ 0 \leq \abs{\phi(x)-\phi(\overline{x})} \leq \theta \abs{x-\overline{x}} \]
    e quindi per il teorema dei 2 carabinieri
    \[ \abs{\phi(x)-\phi(\overline{x})} \rightarrow 0, \, x \rightarrow \overline{x} \]
    che è equivalente a dire che
    \[ \phi(x) \rightarrow \phi(\bar{x}),\, x \rightarrow \bar{x} \]
\end{enumerate}
\begin{proof}[\unskip\nopunct]
\textbf{Dimostrazione}\\
Cominciamo dimostrando l'$\exists$ di un punto fisso, limitandoci al caso $[a,b]$ limitato: in questo caso basta l'ipotesi $(1)$ e la continuità di $\phi$ (non serve che $\phi$ sia una contrazione). \\
Siccome $\phi$ è continua, tale è 
\[ f(x) = x - \phi(x) \]
Se $a = \phi(a)$ oppure $b = \phi(b)$ allora $a$ oppure $b$ sono punto fisso.\\
Se invece $a \neq \phi(a)$ e $b \neq \phi(b)$ siccome $a \leq \phi(x) \leq b \ \forall x \in [a,b]$ si ha $a - \phi(a) < 0$ e $b - \phi(b) > 0$ cioè $f$ è continua e cambia segno agli estremi 
\[\Longrightarrow \exists\, \xi \in (a,b):f(\xi)=0\] 
cioè \[\exists\, \xi \in (a,b) : \xi = \phi(\xi)\]
La continuità non basta però a garantire l'unicità del punto fisso, come si vede da questo disegno
\begin{center}
    \includegraphics[width=0.5\textwidth]{img1_pag7.png}
\end{center}
Ma se $\phi$ è una contrazione, l'unicità è assicurata. \\
Infatti se $\exists \xi_1, \xi_2 \in I$ con $\xi_1 \neq \xi_2$ tali che $\xi_1 = \phi(\xi_1)$ e $\xi_2 = \phi(\xi_2)$ allora 
\[\abs{\xi_1 - \xi_2} = \abs{\phi(\xi_1) - \phi(\xi_2)} \leq \theta\abs{\xi_1 - \xi_2}\] 
cioè $\theta \geq 1$ contro l'ipotesi che $\theta < 1$. \\
Resta da dimostrare che $\forall x_0 \in I$, definendo $x_{n+1} = \phi(x_n)$, $n \geq 0$ si ha $\lim_{n\to\infty} x_n = \xi$. \\
Ora 
\[e_{n+1} = \abs{x_{n+1} - \xi} = \abs{\phi(x_n) - \phi(\xi)} \leq \theta\abs{x_n - \xi} = \theta e_n\]
da cui 
\[\begin{split}
    e_1 & \leq \theta e_0,\\
    e_2 & \leq \theta e_1 \leq \theta^2 e_0, \\
    & \vdots \\
    e_n & \leq \theta^n e_0 \to 0, \ n\to\infty \text{ perchè } \theta \in (0,1)
\end{split}\]
\end{proof}

\subsection{Stima a priori e a posteriori}
È il caso di fare subito alcune osservazioni importanti: 
\begin{enumerate}
    \item [A)] nel teorema delle contrazioni, la dimostrazione generale è basata sul fatto che la successione $\{ x_n \}$ è di Cauchy e quindi convergente a uno $\xi \in I$ (perché $I$ è chiuso) che è automaticamente punto fisso perché per continuità di $\phi$, 
    \[\xi = \lim x_{n+1} = \lim \phi(x_n) = \phi(\lim x_n) = \phi(\xi)\]
    Ma anche con la dimostrazione scritta sopra, si ottiene la STIMA \uline{A PRIORI} dell'errore \[e_n \leq \theta^n e_0\]
    Se $\theta$ (ed $e_0$) sono noti, questa permette a priori di stabilire il numero di iterazioni sufficiente ad ottenere $\xi$ con una tolleranza $\varepsilon > 0$, risolvendo la disuguaglianza
    \[\theta^n e_0 \leq \varepsilon \Longleftrightarrow e^{n\cdot \log{\theta}} \leq e^{\log{\frac{\varepsilon}{e_0}}} \Longleftrightarrow n \geq \frac{\log{\frac{\varepsilon}{e_0}}}{\log{\theta}} = -\frac{\log{\frac{\varepsilon}{e_0}}}{\abs{\log{\theta}}} = \frac{\log{\frac{e_0}{\varepsilon}}}{\abs{\log{\theta}}}\]
    visto che $\theta \in (0,1)$ e $\log{\theta} < 0$.\\
    Notiamo che la convergenza e la stima ottenuta valgono $\forall x_0 \in I$, cioè le iterazioni di punto fisso che costruiscono (infinite) successioni diverse l'una dall'altra al variare di $x_0$, in ogni caso forniscono successioni che convergono tutte \uline{allo stesso limite} che è l'\uline{unico punto fisso} di $\phi$ in $I$, con una convergenza che è almeno lineare (ordine \uline{almeno} $p=1$) perché $e_{n+1} \leq \theta e_n$
    \item [B)] si può facilmente ottenere una STIMA \uline{A POSTERIORI} dell'errore che spesso è più precisa della stima a priori: basta infatti scrivere
    \[ x_{n+1} - \xi = x_{n+1} - x_n + x_n - \xi \] ma 
    \[x_{n+1}-\xi =\phi(x_n) - \phi(\xi) \underset{\text{VALOR  MEDIO}}{\underset{\uparrow}{=}} \phi'(z_n)(x_n - \xi), \quad z_n \in int(x_n,\xi) \] da cui 
    \[ \phi'(z_n)(x_n - \xi) = x_{n+1}-x_n+x_n-\xi \] ovvero 
    \[ (1-\phi'(z_n))(x_n - \xi) = x_n-x_{n+1} \]
    e passando ai moduli
    \[\frac{\abs{x_{n+1}-x_n}}{\abs{1-\phi'(z_n)}} = e_n \le \frac{\abs{x_{n+1}-x_n}}{1-\theta} \quad \Bigl( 1^\circ \text{ stima a posteriori}\Bigr)\] perchè 
    \[ \abs{\phi'(z_n)} \le \theta < 1 \Longrightarrow \abs{1-\phi'(z_n)} \ge \abs{1-\abs{\phi'(z_n)}} \ge 1 - \theta > 0\] e 
    \[ \frac{1}{\abs{1-\phi'(z_n)}} \le \frac{1}{1-\theta} \]
    In pratica abbiamo fatto vedere che l'errore è stimato dallo STEP $= \abs{x_{n+1}-x_n}$ a meno del fattore $\frac{1}{(1-\theta)}=$ peso.\\
    Se $\theta$ è piccolo, lo step diventa da solo una buona stima dell'errore perchè il peso è $\approx 1$; invece se $\theta$ è vicino ad 1, lo step va corretto per evitare una possibile sottostima.\\
    Ma se $\phi \in C^1(I)$, siccome $z_n \to \xi$ allora $\phi'(z_n) \to \phi'(\xi)$, $n \to \infty$, quindi una stima a posteriori migliore è tendenzialmente la stima empirica (almeno per $n$ abbastanza grande)
    \[e_n = \frac{\abs{x_{n+1}-x_n}}{1-\phi'(z_n)} \approx \frac{\abs{x_{n+1}-x_n}}{1-\phi'(x_n)} \quad \Bigl( 2^\circ \text{ stima a posteriori}\Bigr)\]
\end{enumerate}
Conviene a questo punto fare un esempio

\subsection{Esempio}
Consideriamo l'equazione in forma di punto fisso
\[x = \phi(x) = e^{-\alpha x}, \quad \alpha>0\]
se \uline{$\alpha<1$}, 
\[\abs{\phi'(x)} = \abs{-\alpha e^{-\alpha x}} \le \alpha < 1\]
e quindi $\phi$ contrae le distanze, cioè l'ipotesi (2) del teorema delle contrazioni è soddisfatta con $\theta = \alpha$; d'altra parte, $0<\phi(x) \le 1\, \forall x \in [0,+\infty)$, quindi anche (1) è soddisfatta con $I = [0,+\infty)$.\\
In realtà, siccome $\phi'(x)=-\alpha e^{-\alpha x}<0$, $\phi$ è strettamente decrescente (e positiva), quindi visto che $\phi(0)=1$ e $0<\phi(1)=1-e^{-\alpha}<1$ si ha
\[0 < \phi(x) \le 1 \quad \forall x \in [0,1]\]
cioè $\phi$ è anche una contrazione di $[a,b] = [0,1]$ in se stesso.\\
Allora $\exists$ un unico $\xi$ punto fisso di $\phi$ in $[0,1]$: prendiamo 
\[x_0 = \frac{1}{2} \Longrightarrow e_0 = \abs{x_0 - \xi} \le \frac{1}{2}\]
e la successione $x_{n+1}=\phi(x_n)$, $n \ge 0$ converge a $\xi$ con la stima a priori dell'errore $e_n \le \frac{\alpha^n}{2}$.
\newline \newline
Nel grafico qui sotto mostriamo l'errore effettivo, la stima a priori e la stima a posteriori dello step (non pesato) e dello step pesato da $\frac{1}{(1-\alpha)}$ e da $\frac{1}{(1-\phi'(x_n))}$ con
\[\alpha = 0.2 \Longrightarrow fl(\xi) = 0.8445798674955478\]
\[\alpha = 0.9 \Longrightarrow fl(\xi) = 0.5887032951482605\]
(questi valori sono stati calcolati con il metodo di Newton, si veda Lez.10-Es.2)
\begin{center}
    \includegraphics[width=0.8\textwidth]{pag16}
\end{center}
Si vede chiaramente che la convergenza è lineare, che la stima a priori è una sovrastima, molto distante dall'errore per $\alpha = 0.9$: infatti
\[\frac{e_{n+1}}{e_n} = \abs{\phi'(z_n)} \to \abs{\phi'(\xi)}, \quad n \to \infty \quad \text{cioè} \quad \abs{\phi'(\xi)} = \alpha e^{-\alpha \xi} = L\]
è la costante asintotica, il parametro che effettivamente regola la velocità di convergenza ($\alpha$ è solo una stima) perchè per $n$ abbastanza grande
\[e_{n+1} \approx Le_n\]
Qui per $\alpha=0.9$ si ha $L \approx 0.53$ che è ben minore di $\alpha$, mentre per $\alpha=0.2$ si ha $L \approx 0.17$ che è poco minore di $\alpha$.\\
In effetti la $2^\circ$ stima a posteriori,
\[e_n \approx \frac{\abs{x_{n+1} - x_n}}{(1 - \phi'(x_n))}\]
è una stima aderente dell'errore ma è shiftata in avanti di 1, fenomeno che abbiamo già visto nella stima con lo step nel metodo di Newton, perchè per stimare $e_n$ bisogna essere al passo $n+1$ (step $=\abs{x_{n+1}-x_n}$)
\newline \newline
Dopo aver discusso questo esempio semplice ma significativo, vediamo che anche per le iterazioni di punto fisso vale un risultato di \uline{convergenza locale} (mentre la formulazione generale del teorema delle contrazioni ha
carattere ``globale", visto che $x_0 \in I$ è arbitrario e per la convergenza non è importante che $x_0$ sia vicino a $\xi$).

\subsection{TEOREMA (convergenza LOCALE delle iterazioni di punto fisso)}
\begin{center}
    \fbox{\begin{minipage}[t]{15cm} 
        Sia $\xi$ punto fisso di $\phi \in C^1(I_{\delta}(\xi))$ dove $I_{\delta}(\xi)=[\xi -\delta, \xi +\delta],\, \delta>0$ e sia $\abs{\phi'(\xi)}<1$
        allora
        \[\Rightarrow \exists\, \delta'\le \delta : x_{n+1} = \phi(x_n), \, n\ge 0, \text{ converge a }\xi, \quad \forall\,x_0 \in I_{\delta'}(\xi)\]
    \end{minipage}}
\end{center}
È chiaro il carattere ``locale" di questo risultato, che fornisce
condizioni sufficienti per la convergenza delle iterazioni di punto fisso purchè $x_0$ \uline{sia abbastanza vicino a $\xi$}.
\newline 
\begin{proof}[\unskip\nopunct]
\textbf{Dimostrazione (facoltativa)}\\
Siccome $\phi'$ è continua in $I_{\delta}(\xi)$ tale è $g(x)=\abs{\phi'(x)}-1$, quindi visto che $g(\xi)<0$ per la permanenza del segno $\exists \, \delta'\le \delta$ tale che $g(x)<0$ $\forall x\in I_{\delta}'(\xi)$. Allora $\phi$ è una contrazione in $I_{\delta}'(\xi)$ con 
\[\theta=\max_{x \in I_{\delta}'(\xi)} \abs{\phi'(x)}\]
cioè vale l'ipotesi (2) del teorema delle contrazioni con $I=I_{\delta'}(\xi)$; resta da far vedere che vale (1).\\
Ora, $\forall x \in I_{\delta'}(\xi)$
\[ \abs{\phi(x) - \xi} = \abs{\phi(x) - \phi(\xi)} \leq \theta \abs{x-\xi} \leq \theta \delta' < \delta' \]
cioè $\phi(x) \in I_{\delta'}(\xi)\, \forall x \in I_{\delta'}(\xi)$ e quindi vale anche l'ipotesi (1) del teorema delle contrazioni. \\
Ne consegue che $\forall x_0 \in I_{\delta'}(\xi)$ la successione $x_{n+1} = \phi(x_n), \, n \ge 0$, converge a $\xi$, unico punto fisso di $\phi$ in $I_{\delta'}(\xi)$
\end{proof}

\bigskip
Come per tutti i metodi iterativi, è importante capire quale sia l'\uline{ordine di convergenza} delle iterazioni di punto fisso.\\
Abbiamo già osservato che nel caso di una contrazione l'ordine è almeno $p=1$ perché vale \[e_{n+1}\leq\theta e_n \quad \text{con} \quad \theta \in (0,1)\]
D'altra parte, nell'esempio svolto prima, abbiamo fatto vedere che l'ordine è esattamente $p=1$ se $\phi'(\xi)\neq 0$ con costante asintotica $L=\abs{\phi'(\xi)}$\\
Diamo ora una caratterizzazione completa col seguente:

\subsection{TEOREMA (ordine di convergenza delle iterazioni di punto fisso)}
\begin{center}
    \fbox{\begin{minipage}[t]{16cm}%
        Sia $\xi$ punto fisso di $\phi \in C^p(I),\ p \geq 1$ 
        con $I$ intervallo di $\mathbb{R}$ e supponiamo di essere in ipotesi che garantiscano la convergenza a $\xi$ di $x_{n+1}=\phi(x_n),\ n \geq 0,$ con $x_0 \in I$ (ad esempio le ipotesi del teorema delle contrazioni) \\
        Allora:
        \begin{enumerate}
            \item $\{x_n\}$ ha ordine esattamente $p=1 \ \iff 0<\abs{\phi'(\xi)}<1$
            \item $\{x_n\}$ ha ordine esattamente $p>1 \ \iff \phi^{(j)}(\xi)=0, \ 1\leq j \leq p-1 \ e \ \phi^{(p)}(\xi)\neq 0$
        \end{enumerate}
    \end{minipage}}
\end{center}
\begin{proof}[\unskip\nopunct]
\uline{Dimostrazione}
\begin{enumerate}
    \item si dimostra subito visto che 
        \[e_{n+1}=\abs{\phi'(z_n)}e_n, \ z_n \in int(\xi,x_n)\]
        per il teorema del valor medio, quindi 
        \[\lim_{n\to\infty} \frac{e_{n+1}}{e_n} = \abs{\phi'(\lim z_n)}=\abs{\phi'(\xi)}\]
    \item per 2) utilizziamo la formula di Taylor di grado $p-1$ centrata in $\xi$ con resto $p$-esimo in forma di Lagrange \\
\[x_{n+1}=\phi(x_n)=\underset{\xi}{\underset{\shortparallel}{\phi(\xi)}}+ \phi'(\xi)(x_n-\xi)+\frac{\phi''(\xi)}{2}(x_n-\xi)^2+\dotso+\frac{\phi^{(p-1)}(\xi)}{(p-1)!}(x_n-\xi)^{p-1}+\frac{\phi^{(p)}(u_n)}{p!}(x_n-\xi)^p\]
con $u_n \in int(\xi,x_n)$\\
\begin{itemize}
\item \textbf{Dimostriamo prima $``\Leftarrow"$ (condizione sufficiente)}\\
se $\phi^{(j)}(\xi)=0, \ 1\leq j \leq p-1$ e $\phi^{(p)}(\xi)\neq 0$, da Taylor 
\[x_{n+1}-\xi=\frac{\phi^{(p)}(u_n)}{p!}(x_n-\xi)^p\]
e passando ai moduli
\[\frac{e_{n+1}}{e_n^p}=\frac{\abs{\phi^{(p)}(u_n)}}{p!}\underset{n\to\infty}{\longrightarrow}\frac{\abs{\phi^{(p)}(\xi)}}{p!}\neq 0\]
perché $u_n\rightarrow \xi$, $n\to\infty$ e $\phi^{(p)}$ è continua quindi 
\[\lim \phi^{(p)}(u_n)=\phi^{(p)}(\lim u_n)=\phi^{(p)}(\xi)\]
ovvero $\{x_n\}$ ha ordine esattamente $p$. \\
\item Per \textbf{dimostrare ``$\Rightarrow$" (condizione necessaria)}\\ 
supponiamo per assurdo che $\{ x_n \}$ abbia ordine esattamente $p$ ma che $\exists\ j<p$ tale che $\phi^{(j)}(\xi)\neq 0 $, prendiamo $k=\min\{ j<p:\phi^{(j)}(\xi)\neq0\}$ e scriviamo 
\[\frac{e_{n+1}}{e_n^p} = \frac{e_{n+1}}{e_n^k}\cdot e_n^{k-p}\]
Ora per ipotesi
\[\frac{e_{n+1}}{e_n^p}\rightarrow L \neq 0 \]
d'altra parte con lo stesso ragionamento usato per ``$\Leftarrow$" tramite la formula di Taylor si avrebbe
\[\frac{e_{n+1}}{e_n^k}\to \frac{\abs{\phi^{(k)}(\xi)}}{k!}=L'\neq 0\] 
ma allora
\[\frac{e_{n+1}}{e_n^p} = \frac{e_{n+1}}{e_n^k}\cdot e_n^{k-p}\]
\[\left( \frac{e_{n+1}}{e_n^k} \to L' \text{ ed } e_n^{k-p} \to \infty \text{ perchè } k-p<0 \text{ ed } e_n \to 0 \right) \]
cioè 
\[\frac{e_{n+1}}{e_n^p} \to \infty, \quad n \to \infty\]
contraddicendo l'ipotesi che abbia limite finito.
\end{itemize}
\end{enumerate}
\end{proof}
\bigskip
Ribadiamo che la condizione data è \uline{NECESSARIA} e \uline{SUFFICIENTE}, cioè fornisce una caratterizzazione completa di quando le iterazioni di punto fisso hanno ordine $p \geq 1$.\\
In particolare (e questo ci servirà tra poco) le iterazioni di punto fisso possono avere convergenza \uline{quadratica} ($\phi'(\xi)=0$ e $\phi''(\xi)\ne 0$), \uline{cubica} ($\phi'(\xi)=\phi''(\xi)=0$ e $\phi'''(\xi)\ne0, \dotso$).

\subsection{Metodo di Newton come iterazione di punto fisso}
Per concludere la lezione, mostriamo infatti che il metodo di Newton si può re-interpretare come iterazione di punto fisso e che, in base a quanto visto sopra, si vede subito che ammette convergenza locale e che la convergenza è quadratica per zeri semplici.\\
Infatti l'iterazione di Newton
\[x_{n+1} = x_n - \frac{f(x_n)}{f'(x_n)}, \quad n\ge 0\]
è di tipo punto fisso con
\[\phi(x) = x - \frac{f(x)}{f'(x)}\]
e se $f \in C^k(I)$ con $f'(x)\ne 0 \,\forall x \in I$ intervallo, allora $\phi \in C^{k-1}(I)$.\\
È evidente che $\xi$ è zero (semplice) di $f \iff \xi$ è punto fisso di $\phi$.\\
Inoltre, il teorema di convergenza \uline{locale} per \uline{Newton} (zeri semplici) discende immediatamente dal teorema di convergenza \uline{locale} per le \uline{iterazioni di punto fisso} se $f \in C^2(I_\delta(\xi))$, infatti
\[ \phi'(x) = \frac{d}{dx}\biggl(x-\frac{f}{f'}\biggr) = 1 - \underbrace{\frac{(f')^2 - ff''}{(f')^2}}_{\text{derivata\, del\,rapporto}} = \frac{ff''}{(f')^2}\]
quindi $f(\xi)=0 \Rightarrow \phi'(\xi)=0$ e $\abs{\phi'(\xi)}=0<1$, allora $\exists\, \delta'\le \delta$ tale che l'iterazione di Newton converge come iterazione di punto fisso $\forall x_0 \in I_{\delta'}(\xi)$.\\
D'altra parte, sempre interpretando Newton come iterazione di punto fisso, è immediato che la convergenza per zeri semplici è almeno quadratica perché $\phi'(\xi)=0$ ed è esattamente quadratica se $f''(\xi)\neq0$, utilizzando la caratterizzazione vista sopra (che però richiede $\phi \in C^2$ e quindi $f\in C^3$) infatti:
\[\phi''=\frac{d}{dx}\left(\frac{f f''}{(f')^2}\right)= \underbrace{\frac{\overset{ \downarrow \ derivata \ del \ prodotto}{(f' f''+f f''')(f')^2-2 f' f'' (f f'')}}{(f')^4}}_{\text{ancora\, derivata\, del\, rapporto}}\]
da cui 
\[ \phi''(\xi)=\frac{f''(\xi)}{f'(\xi)}\neq 0 \]
perché 
\begin{equation*}
    f''(\xi)\neq 0 \ ( e \ f'(\xi)\neq0)
\end{equation*}
Non è difficile vedere (come approfondimento facoltativo) che interpretando Newton come iterazione di punto fisso, nel caso di zero multiplo l'ordine di convergenza diventa $p=1$ con costante asintotica 
\[\abs{\phi'(\xi)}=1-\frac{1}{m}\] 
dove $m$ è la molteplicità di $\xi$ (il numero di derivate successive che si annullano in $\xi$) e che se $m$ è nota, l'iterazione
\[x_{n+1}=x_n-m\frac{f(x_n)}{f'(x_n)}=\phi_m(x_n)\]
torna di ordine almeno $p=2$ perché 
\[\phi'_m(\xi)=\lim_{x\to\xi}\phi'_m(x)=0\]
\end{document}